\documentclass[11pt]{article} % use larger type; default would be 10pt

\usepackage[utf8]{inputenc} % set input encoding (not needed with XeLaTeX)

\usepackage{geometry} % to change the page dimensions
\geometry{a4paper} % or letterpaper (US) or a5paper or....
% \geometry{margin=2in} % for example, change the margins to 2 inches all round
% \geometry{landscape} % set up the page for landscape
%   read geometry.pdf for detailed page layout information

\usepackage{graphicx} % support the \includegraphics command and options
\usepackage{authblk}
\usepackage{url}

% \usepackage[parfill]{parskip} % Activate to begin paragraphs with an empty line rather than an indent

%%% PACKAGES
\usepackage{booktabs} % for much better looking tables
\usepackage{array} % for better arrays (eg matrices) in maths
\usepackage{paralist} % very flexible & customisable lists (eg. enumerate/itemize, etc.)
\usepackage{verbatim} % adds environment for commenting out blocks of text & for better verbatim
\usepackage{subfig} % make it possible to include more than one captioned figure/table in a single float
% These packages are all incorporated in the memoir class to one degree or another...

\usepackage[polish]{babel}	
\usepackage{polski}
\usepackage[T1]{fontenc}
\frenchspacing	

\usepackage{indentfirst}

\usepackage{fancyhdr} % This should be set AFTER setting up the page geometry
\pagestyle{fancy} % options: empty , plain , fancy
\renewcommand{\headrulewidth}{0pt} % customise the layout...
\lhead{}\chead{}\rhead{}
\lfoot{}\cfoot{\thepage}\rfoot{}

%%% SECTION TITLE APPEARANCE
\usepackage{sectsty}
\allsectionsfont{\sffamily\mdseries\upshape} % (See the fntguide.pdf for font help)
% (This matches ConTeXt defaults)

%%% ToC (table of contents) APPEARANCE
\usepackage[nottoc,notlof,notlot]{tocbibind} % Put the bibliography in the ToC
\usepackage[titles,subfigure]{tocloft} % Alter the style of the Table of Contents
\renewcommand{\cftsecfont}{\rmfamily\mdseries\upshape}
\renewcommand{\cftsecpagefont}{\rmfamily\mdseries\upshape} % No bold!

%%% END Article customizations

%%% The "real" document content comes below...

\title{San Francisco Crime Classification\\ \large{Kaggle competition  }}
\author{Łukasz Rados, Wojciech Kusa}

\affil{Wydział Fizyki i Informatyki Stosowanej \\ Akademia Górniczo-Hutnicza w Krakowie}


%\date{} % Activate to display a given date or no date (if empty),
         % otherwise the current date is printed 

\begin{document}
\maketitle

\section{Wprowadzenie}

Celem projektu było stworzenie oprogramowania pozwalającego dokonać klasyfikacji przestępstw na podstawie danych czasoprzestrzennych z raportów policyjnych dla miasta San Francisco w Stanach Zjednoczonych. 


\section{Dane}

Zbiór danych zawiera incydenty zgłoszone policji w San Francisco pomiędzy 01.01.2003r. a 13.05.2015r.. Podzielony jest na dwie podgrupy (prawie równoliczne, w każdej po około 850 tysięcy elementów) :
\begin{itemize}
\item zbiór treningowy - zawierający zgłoszenia z tygodni parzystych,

\item zbiór testowy -  zawierający zgłoszenia z tygodni nieparzystych.
\end{itemize} 

Przykładowe wiersze danych treningowych znajdują się na Rysunku \ref{fig:train_data}. Dane składają się z następujących pól:

\begin{itemize}
\item Dates - znacznik czasu przestępstwa
\item DayOfWeek - dzień tygodnia
\item PdDistrict - nazwa departamentu policji odbierającego zgłoszenie
\item Address - przybliżony adres przestępstwa
\item X - długość geograficzna
\item Y - szerokość geograficzna
\item Category - kategoria przestępstwa (tylko dla zbioru treingowego). Zmienna, którą należało przewidzieć
\item Descript - szczegółowy opis przestępstwa (tylko dla zbioru treingowego)
\item Resolution - jaki był wynik działania policji (tylko dla zbioru treingowego)

\end{itemize}

\begin{figure}[!h]
  \centering
    \includegraphics[width=0.95\textwidth]{images/train_data.png}
  \caption{Przykładowe dane treningowe. Źródło: \protect\url{https://www.kaggle.com/c/sf-crime/data} }
\end{figure}



\subsection{Wstępna analiza zbioru treningowego}
 tutaj rysunki

\subsection{Zastosowane deskryptory}


\section{Zastosowane algorytmy}
\subsection{Lasy losowe - Random Forest}

\subsection{Generalized Linear Model}


\section{Implementacja}

kilka słów o użytych bibliotekach

\section{Uzyskane wyniki}




\section{Podsumowanie}



\end{document}
